\documentclass[a4paper, 12pt]{extreport}

% Под pdflatex
\usepackage{cmap} 
\usepackage[utf8]{inputenc}
\usepackage[T2A]{fontenc}
\usepackage[english,russian]{babel}

\babelprovide[hyphenrules=russian]{russian}
% Под lualatex \usepackage{fontspec}
%\usepackage{polyglossia}
%\setmainlanguage{russian}
%\setotherlanguage{english}

% Шрифты с кириллицей
%\setmainfont{CMU Serif}[
%Extension=.ttf,
%UprightFont=*-Regular,
%BoldFont=*-Bold,
%ItalicFont=*-Italic,
%BoldItalicFont=*-BoldItalic
%]
%\setsansfont{CMU Sans}
%\setmonofont{CMU Mono}

%\usepackage{amssymb,amsfonts,amsmath,mathtext,cite,enumerate,float}
%\usepackage{glyphtounicode}
%\pdfgentounicode=1
%\usepackage{glyphtounicode}
%\pdfgentounicode=1
\usepackage{amssymb,amsfonts,amsmath,cite,enumerate,float}

% Векторный шрифт
%\IfFileExists{tempora.sty}{%
%	\usepackage{tempora}%
%}{
%	\usepackage{mathptmx}%
%}
% убрал, потому что скачал локально cm-super, который подставляется автоматически
\renewcommand{\rmdefault}{cmr}

\usepackage{pgfplots}
\usepackage{graphicx}
\usepackage{tocloft}
\usepackage{listings}
\usepackage[justification=centering]{caption}
\captionsetup[table]{justification=centering,singlelinecheck=false}
%\usepackage{titlesec}
\usepackage{setspace}
\usepackage{geometry}
\usepackage{indentfirst}
\usepackage{pdfpages}
\usepackage{enumerate,letltxmacro}
%\usepackage{threeparttable}
\usepackage{hyperref}
\usepackage{flafter}
\usepackage{enumitem}
%\usepackage{multirow}
\usepackage{ragged2e}

%\usepackage[figure,table]{totalcount}
%\usepackage{lastpage}

\usepackage{longtable}
\usepackage{array}
\usepackage{geometry}
\usepackage[table]{xcolor}
\usepackage{colortbl}
%\usepackage[figure,table,enumiv]{totalcount}
\usepackage{booktabs}

\usepackage{subcaption}

%\usepackage{placeins}


\setlist{nosep}

\newcommand{\ssr}[1]{\begin{center}
    \LARGE\bfseries{#1}
  \end{center} \addcontentsline{toc}{chapter}{#1}  }

\makeatletter
\renewcommand\LARGE{\@setfontsize\LARGE{22pt}{20}}
\renewcommand\Large{\@setfontsize\Large{20pt}{20}}
\renewcommand\large{\@setfontsize\large{16pt}{20}}
\makeatother
\RequirePackage{titlesec}
\titleformat{\chapter}[block]{\hspace{\parindent}\large\bfseries}{\thechapter}{0.5em}{\large\bfseries\raggedright}
\titleformat{name=\chapter,numberless}[block]{\hspace{\parindent}}{}{0pt}{\large\bfseries\centering}
\titleformat{\section}[block]{\hspace{\parindent}\large\bfseries}{\thesection}{0.5em}{\large\bfseries\raggedright}
\titleformat{\subsection}[block]{\hspace{\parindent}\large\bfseries}{\thesubsection}{0.5em}{\large\bfseries\raggedright}
\titleformat{\subsubsection}[block]{\hspace{\parindent}\large\bfseries}{\thesubsection}{0.5em}{\large\bfseries\raggedright}
\titlespacing{\chapter}{12.5mm}{-22pt}{10pt}
\titlespacing{\section}{12.5mm}{10pt}{10pt}
\titlespacing{\subsection}{12.5mm}{10pt}{10pt}
\titlespacing{\subsubsection}{12.5mm}{10pt}{10pt}

\makeatletter
\renewcommand{\@biblabel}[1]{#1.}
\makeatother

\geometry{left=30mm}
\geometry{right=10mm}
\geometry{top=20mm}
\geometry{bottom=20mm}

\onehalfspacing

\renewcommand{\theenumi}{\arabic{enumi}}
\renewcommand{\labelenumi}{\arabic{enumi}\text{)}}
\renewcommand{\theenumii}{.\arabic{enumii}}
\renewcommand{\labelenumii}{\asbuk{enumii}\text{)}}
\renewcommand{\theenumiii}{.\arabic{enumiii}}
\renewcommand{\labelenumiii}{\arabic{enumi}.\arabic{enumii}.\arabic{enumiii}.}

\renewcommand{\cftchapleader}{\cftdotfill{\cftdotsep}}

\addto\captionsrussian{\renewcommand{\figurename}{Рисунок}}
\DeclareCaptionLabelSeparator{dash}{~---~}
\captionsetup{labelsep=dash}

\captionsetup[figure]{justification=centering,labelsep=dash}

\graphicspath{{images/}}%путь к рисункам

\newcommand{\floor}[1]{\lfloor #1 \rfloor}

\renewcommand{\lstlistingname}{Листинг}


\DeclareCaptionFormat{listing}{\raggedright#1#2#3}
\captionsetup[lstlisting]{format=listing,singlelinecheck=false,justification=raggedright}

\lstset{ %
    basicstyle=\small\sffamily,
    resetmargins=true,
%    language=rust,
    numbers=left,                % где поставить нумерацию строк (слева\справа)
    numberstyle=\tiny,           % размер шрифта для номеров строк
    stepnumber=1,                % размер шага между двумя номерами строк
    numbersep=5pt,               % как далеко отстоят номера строк от подсвечиваемого кода
    showspaces=false,            % показывать или нет пробелы специальными отступами
    showstringspaces=false,      % показывать или нет пробелы в строках
    showtabs=false,              % показывать или нет табуляцию в строках
    frame=single,                % рисовать рамку вокруг кода
    tabsize=2,                   % размер табуляции по умолчанию равен 2 пробелам
    captionpos=t,                % позиция заголовка вверху [t] или внизу [b] 
    breaklines=true,             % автоматически переносить строки (да\нет)
    breakatwhitespace=false,     % переносить строки только если есть пробел
    escapeinside={\#*}{*)},      % если нужно добавить комментарии в коде
    keepspaces=true
}

\pgfplotsset{width=0.85\linewidth, height=0.5\columnwidth}

\linespread{1.3}

\parindent=1.25cm

\def\labelitemi{---}
\setlist[itemize]{
    leftmargin=\parindent,
    itemindent=1.8em,
    labelsep=-0.3em,
    align=left,
    parsep=0pt,
    itemsep=0pt,
    topsep=0pt,
    before={\setlength{\parindent}{12.5mm}}
}
\setlist[enumerate]{leftmargin=1.25cm, itemindent=0.55cm}

\newcommand{\specialcell}[2][c]{%
    \begin{tabular}[#1]{@{}c@{}}#2\end{tabular}}

\frenchspacing

\usepackage{hyperref}
\hypersetup{
    colorlinks=false,
    linkcolor=blue,
    filecolor=magenta,
    urlcolor=cyan,
    pdftitle={Лабораторная работа                                                                                    },
    pdfpagemode=FullScreen,
}

\DeclareMathOperator*{\argmax}{arg\,max}

%%\captionsetup[listing]{justification=raggedright}
\captionsetup[]{justification=raggedright}

\titleformat{\part}[block]
{\large\bfseries}{\hspace{12.5mm}\thechapter}{0.5em}{\large\centering}

\titleformat{\chapter}[block]
{\large\bfseries}{\hspace{12.5mm}\thechapter}{0.5em}{\large\raggedright}

\titleformat{\section}[block]
{\large\bfseries}{\hspace{12.5mm}\thesection}{0.5em}{\large\raggedright}
\renewcommand{\thesection}{\arabic{chapter}.\arabic{section}.}

\titleformat{\subsection}[block]
{\large\bfseries}{\hspace{12.5mm}\thesubsection}{0.5em}{\large\raggedright}
\renewcommand{\thesubsection}{\arabic{chapter}.\arabic{section}.\arabic{subsection}.}

\titleformat{\subsubsection}[block]
{\large\bfseries}{\hspace{12.5mm}\thesubsubsection}{0.5em}{\large\raggedright}
\renewcommand{\thesubsubsection}{\arabic{chapter}.\arabic{section}.\arabic{subsection}.\arabic{subsubsection}.}

% Формат для ненумерованных разделов (section*)
\titleformat{name=\section,numberless}[block]
{\large\bfseries}{\hspace{12.5mm}}{0em}{\large\raggedright}

% Формат для ненумерованных подразделов (subsection*)
\titleformat{name=\subsection,numberless}[block]
{\large\bfseries}{\hspace{12.5mm}}{0em}{\large\raggedright}

% Формат для ненумерованных подразделов (subsubsection*)
\titleformat{name=\subsubsection,numberless}[block]
{\large\bfseries}{\hspace{12.5mm}}{0em}{\large\raggedright}

\renewcommand{\thesubfigure}{\asbuk{subfigure}}
\renewcommand{\cfttoctitlefont}{\LARGE\bfseries\centering}



\lefthyphenmin=2
\righthyphenmin=2
\tolerance=2000
\emergencystretch=10pt
\hyphenpenalty=50
\exhyphenpenalty=50

\usepackage{listings}
\usepackage{xcolor}

\lstdefinestyle{mydefault}{
    frame=single,
    xleftmargin=0.4cm,
    framexleftmargin=-0.3cm,
    framesep=0.3cm,
    linewidth=\dimexpr\linewidth-0.7cm\relax,
    basicstyle=\ttfamily\small,
    breaklines=true,
    tabsize=2
}

\lstnewenvironment{mylisting}[3][]{
    \lstset{
        style=mydefault,
        caption={#2},
        label={#3},
        #1
    }
}{}

% для автоматизации реферата
\usepackage{lastpage}
\usepackage[figure,table,enumiv]{totalcount}

% \begin{figure}[H]
%     \centering
%     \rotatebox{90}{
%         \begin{minipage}{0.9\textheight}
%             \centering
%             \includegraphics[width=0.95\linewidth]{RendererCore.pdf}
%             \captionof{figure}{Диаграмма классов домена отрисовки, ч. 1}
%             \label{fig:RendererCore}
%         \end{minipage}
%     }
% \end{figure} % пример вставки картинки боком
